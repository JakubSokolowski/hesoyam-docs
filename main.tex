% University Assignment Title Page 
% LaTeX Template
% Version 1.0 (27/12/12)
%
% This template has been downloaded from:
% http://www.LaTeXTemplates.com
%
% Original author:
% WikiBooks (http://en.wikibooks.org/wiki/LaTeX/Title_Creation)
%
% License:
% CC BY-NC-SA 3.0 (http://creativecommons.org/licenses/by-nc-sa/3.0/)
% 
% Instructions for using this template:
% This title page is capable of being compiled as is. This is not useful for 
% including it in another document. To do this, you have two options: 
%
% 1) Copy/paste everything between \begin{document} and \end{document} 
% starting at \begin{titlepage} and paste this into another LaTeX file where you 
% want your title page.
% OR
% 2) Remove everything outside the \begin{titlepage} and \end{titlepage} and 
% move this file to the same directory as the LaTeX file you wish to add it to. 
% Then add \input{./title_page_1.tex} to your LaTeX file where you want your
% title page.
%
%%%%%%%%%%%%%%%%%%%%%%%%%%%%%%%%%%%%%%%%%
%\title{Title page with logo}
%----------------------------------------------------------------------------------------
% PACKAGES AND OTHER DOCUMENT CONFIGURATIONS
%----------------------------------------------------------------------------------------
\documentclass[12pt]{article}
\usepackage{polski}
\usepackage[utf8]{inputenc}
\usepackage{amsmath}
\usepackage{mathtools}
\usepackage{hyperref}
\usepackage{tikz}
\tikzset{
heap/.style={
every node/.style={shape=rectangle,rounded corners, draw},
level 1/.style={sibling distance=30mm},
level 2/.style={sibling distance=10mm}
}
}
\usepackage{blkarray, bigstrut}
\usepackage{graphicx}
\usepackage[colorinlistoftodos]{todonotes}
\usepackage[left=2.5cm,top=3cm,right=2.5cm,bottom=3cm,bindingoffset=0.5cm]{geometry}
\usepackage{multicol}
\usepackage{afterpage}
\usepackage{array}
\usepackage{pgfplots}
\usepackage{listings}
\lstset{
  showstringspaces=false
basicstyle=\ttfamily,
columns=fullflexible,
frame=single,
breaklines=true,
postbreak=\mbox{\textcolor{red}{$\hookrightarrow$}\space},
escapeinside={(*@}{@*)},
}
\usepackage{tabularx}
\usepackage{listings}
\usepackage{dirtree}
\usepackage{caption}
\usepackage[section]{placeins}
\usepackage{amsfonts}
%\captionsetup[figure]{font=small,labelfont=small}
%\captionsetup[table]{font=small,labelfont=small}

\newcommand\blankpage{%
\null
\thispagestyle{empty}%
\addtocounter{page}{-1}%
\newpage}
\pgfplotsset{compat=1.15}
\setcounter{tocdepth}{3}
\setcounter{secnumdepth}{3}
\setlength{\marginparwidth}{2cm}
\begin{document}
\begin{titlepage}

\newcommand{\HRule}{\rule{\linewidth}{0.5mm}} % Defines a new command for the horizontal lines, change thickness here

\center % Center everything on the page
\vspace*{\fill}
%----------------------------------------------------------------------------------------
% HEADING SECTIONS
%----------------------------------------------------------------------------------------

\textsc{\LARGE Politechnika Wrocławska}\\[0.5cm] % Name of your university/college
\textsc{\Large Wydział Elektroniki}\\[1.5cm] % Major heading such as course name
\textsc{\large Inżynieria e-systemów - technologia Java}\\[0.5cm] % Minor heading such as course title

%----------------------------------------------------------------------------------------
% TITLE SECTION
%----------------------------------------------------------------------------------------

\HRule \\[0.4cm]
{ \huge \bfseries Aplikacja internetowa służąca do analizowania sentymentu na rynkach finansowych}\\[0.4cm] % Title of your document
\HRule \\[1.5cm]
%----------------------------------------------------------------------------------------
% AUTHOR SECTION
%----------------------------------------------------------------------------------------

\begin{minipage}[t]{0.4\textwidth}
\begin{flushleft} \large
\emph{Autorzy:}\\
Jakub Sokołowski 226080\\
Konrad Olszewski 238898\\
\end{flushleft}
\end{minipage}
~
\begin{minipage}[t]{0.5\textwidth}
\begin{flushright} \large
\emph{Prowadzący:} \\
Dr inż. Tomasz Walkowiak
\end{flushright}
\end{minipage}\\

% If you don't want a supervisor, uncomment the two lines below and remove the section above
%\Large \emph{Author:}\\
%John \textsc{Smith}\\[3cm] % Your name

%----------------------------------------------------------------------------------------
% DATE SECTION
%----------------------------------------------------------------------------------------
\vspace*{\fill}

{\large \today}\\[2cm] % Date, change the \today to a set date if you want to be precise
\end{titlepage}
\newpage
\tableofcontents
\newpage
\section{Cel projektu}
Celem projektu jest stworzenie aplikacji webowej umożliwiającej analizę sentymentu na rynkach finansowych.
Przez sentyment rozumiemy: opinie na temat sytuacji na rynku (uzyskiwane za pomocą analizy postów, artykułów oraz komentarzy).
\subsection{Sentyment na rynkach finansowych}
W finansach behawioralnych, ważnym wskaźnikiem pozwalającym na opisanie zachowanie inwestorów lub osób zainteresowanych rynkiem jest sentyment. Sentyment możemy ogólnie zdefiniować jako skłonność do spekulacji (podejmowania nadmiernego ryzyka) oraz system oczekiwań, co do przyszłego kursu danego instrumentu finansowego. W pracy \textit{"Investor Sentiment in the Stock Market"} Baker i Wurgler rozważają teoretyczne wpływy sentymentu na rynki oraz instrumenty finansowe. Zgodnie z obserwacjami autorów badania, instrumenty finansowe cechujące się zwiększoną zmiennością cen oraz skomplikowaną wyceną, charakteryzują się większą podatnością na sentyment. \textbf{TUTAJ OPISZEMY W JAKI SPOSÓB ANALIZOWANE SĄ POSTY}. 
\section{Wymagania i założenia projektu}
\subsection{Wymagania funkcjonalne}
\textbf{TO SAMO TYLKO DŁUŻEJ. MOŻE JAKIEŚ PRZYPADKI UŻYCIA DODAĆ}
\begin{enumerate}
  \item {Wizualizacja przebiegu czasowego ceny instrumentu, za pomocą wykresów "candlestick".}
  \item {Wyświetlanie danych sentymentu z zadanego przedziały czasowego filtrowanie i agregacja danych.}
  \item {Umożliwienie użytkownikowi uzupełnienia luk w danych, z wykorzystaniem komputera użytkownika.}
  \item {Umożliwienie pobrania danych w różnych formatach.}
\end{enumerate}
\subsection{Wymagania niefunkcjonalne}
\begin{enumerate}
  \item {System przewiduje maksymalną ilość użytkowników równocześnie korzystających z systemu na poziomie 100 osób.}
  \item {System okresowo (godziny nocne, z uwagi na mniejsze obciążenie serwera) generuje kopie zapasowe.}
  \item {System przewiduje współpracę jedynie z nowoczesnymi przeglądarkami internetowymi.}
\end{enumerate}
\clearpage
\subsection{Założenia}
\textbf{DANE FINANSOWE SĄ DROGIE WIĘC BIERZEMY DANE Z CRYPTO. DANE ZBIERAMY Z TWITTERA I REDDITA}
\section{Architektura projektu}
\subsection{Model danych}
\subsection{Moduły}
\subsection{Technologie}
\subsubsection{Stack technologiczny}
\begin{itemize}
  \item {\textbf{aplikacja klienta}: Angular7}
  \item {\textbf{komunikacja klient-serwer}: REST API, WebSockets}
  \item {\textbf{serwer}: javaEE}
  \item {\textbf{baza danych}: mongoDB}
\end{itemize}
\subsubsection{Wybrane technologie}
\begin{itemize}
    \item {\textbf{MongoDB} - jest nierelacyjnym systemem zarządzania bazą danych, który charakteryzuje się wysokiego stopnia skalowalnością oraz brakiem jednolitej struktury obsługiwania baz danych. MongoDB składuje dokumenty w postaci plików JSON, które zwyczajowo nazywane są dokumentami i umieszczane są w tzw. kolekcjach. Silnik nie wymaga tworzenie schematu bazy danych - schemat działania opiera się na wstawianiu dokumentów w odpowiednie kolekcje. W przypadku, gdy dana kolekcja nie istnieje, zostaje ona automatycznie stworzona. W porównaniu z relacyjnymi bazami danych, MongoDB charakteryzuje się większą wydajnością oraz możliwością skalowania bazy na wiele serwerów.}
    \item {\textbf{Angular7} - }
    \item {\textbf{REST API, WebSockets} - }
\end{itemize} Journal of Economic Perspectives, 21 (2): 129-152.

\begin{thebibliography}{9}
\bibitem{sentiment} 
Baker, Malcolm, and Jeffrey Wurgler. 2007. 
\textit{"Investor Sentiment in the Stock Market."}
 Journal of Economic Perspectives, 21 (2): 129-152.
\end{thebibliography}
\end{document}